
    \documentclass{article}
    \usepackage{amsmath}
    \usepackage{amssymb}
    \usepackage{amsfonts}
    \usepackage{amsthm} 
    
    \begin{document}
    \textbf{Proof:}

Given a valid flow in a network G with capacities and V $\in$ V in G, the flow still satisfies the conservation constraints on this network.

Let A be a subset of vertices in G. We want to show that the total flow out of A is equal to the total flow into A.

By summing over all edges in G:

\begin{equation}
\sum_{e} f(e) = \sum_{e} f(e)
\end{equation}

Consider an edge e = (a, b) in G.

We rearrange the equation:

\begin{equation}
\sum_{e \in \delta^{+}(A)} f(e) = \sum_{e \in \delta^{-}(A)} f(e)
\end{equation}

Where $\delta^{+}(A)$ represents the set of edges with one end in A, and $\delta^{-}(A)$ represents the set of edges with one end in A.

Now, let's consider the different cases for the edge e in terms of its endpoints:

\begin{itemize}
\item Case 1: Entirely in A, so both the start and end are in A.
\item Case 2: Leaving A, so it starts at a, where a $\in$ A, and goes towards b, where b $\in$ $V \textbackslash A$.
\item Case 3: Entering A, so it starts at b, where b $\in$ $V \textbackslash A$, and comes towards a, where a $\in$ A.
\item Case 4: Entirely in $V \textbackslash A$, so it never appears in the sum.
\end{itemize}

From observation, if edge e = (a, b) is in Case 1, then $f(e)$ appears on the left-hand side when $v = b$, and $f(e)$ also appears on the right-hand side when $v = a$. 

Thus, we have:

\begin{equation}
\sum_{e \in \delta^{+}(A)} f(e) + \sum_{e \in \delta^{-}(A)} f(e) = \sum_{e \in \delta^{+}(A)} f(e)
\end{equation}

This implies that the total flow out of A equals the total flow into A, which completes the proof.
    \end{document}
    